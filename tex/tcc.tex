\documentclass[12pt]{report}
\usepackage[utf8]{inputenc}
\usepackage[T1]{fontenc}
\usepackage[portuguese, brazil, english]{babel}
\usepackage{graphics}
\usepackage{url}
\usepackage{lipsum}
\usepackage{graphicx}
\usepackage{geometry}
\usepackage{amssymb}
\usepackage{float}
\usepackage{verbatim} 
\usepackage{amsmath} 
\usepackage{amsfonts}
\usepackage{amssymb}
\usepackage{lmodern}
\usepackage[portuguese,ruled,lined]{algorithm2e}
\usepackage{algorithmic}
\usepackage{caption}
\usepackage{subcaption}
\usepackage{indentfirst}
\usepackage{mathrsfs,amsmath}
\usepackage{amstext}
\usepackage{hyperref}
\usepackage{setspace}
\usepackage[refpage]{nomencl}
\usepackage{nomencl}
% \usepackage[alf,abnt-etal-cite=2,abnt-etal-list=0,abnt-etal-text=it,versalete,bibjustif]{abntex2cite}
\usepackage[alf]{abntex2cite}

\usepackage{epstopdf}
\usepackage{datetime}
%\usepackage[acronym]{glossaries}
%\usepackage{glossaries}
\usepackage{acro}
%\usepackage[acronym]{glossaries}
%\usepackage{longtable}
% probably a good idea for the nomenclature entries:
%\acsetup{first-style=short}
% class `abbrev': abbreviations:
%\DeclareAcronym{ny}{
%  short = NY ,
%  long  = New York ,
%  class = abbrev
%}
%\DeclareAcronym{la}{
%  short = LA ,
%  long  = Los Angeles ,
%  class = abbrev
%}
%\enableregime[utf]

\hypersetup{
    %bookmarks=true,         % show bookmarks bar?
    unicode=false,          % non-Latin characters in Acrobat’s bookmarks
    pdftoolbar=true,        % show Acrobat’s toolbar?
    pdfmenubar=true,        % show Acrobat’s menu?
    pdffitwindow=false,     % window fit to page when opened
    pdfstartview={FitH},    % fits the width of the page to the window
    pdftitle={My title},    % title
    pdfauthor={Author},     % author
    pdfsubject={Subject},   % subject of the document
    pdfcreator={Creator},   % creator of the document
    pdfproducer={Producer}, % producer of the document
    pdfkeywords={keyword1, key2, key3}, % list of keywords
    pdfnewwindow=true,      % links in new PDF window
    colorlinks=true,       % false: boxed links; true: colored links
    linkcolor=blue,          % color of internal links (change box color with linkbordercolor)
    citecolor=blue,        % color of links to bibliography
    filecolor=magenta,      % color of file links
    urlcolor=blue           % color of external links
}
\geometry{left=2.5cm, top=2cm, bottom=2.5cm, right=2cm}
\newcommand{\euler}{\textit{e}}
\newcommand{\complexSymbol}{\textit{j}}
\setstretch{1.5}

\hfuzz=30pt
%\vfuzz=20pt
\hbadness=2000
\vbadness=\maxdimen

\def\worktitle{Desenvolvimento de um {\it drum pad} usando visão artificial}
\def\workauthor{Luciano Rodrigues Lucio Neto}
\def\workadvisor{Agostinho de Medeiros Brito Júnior}

\usepackage{afterpage}

\newcommand\blankpage{%
    \null
    \thispagestyle{empty}%
    \addtocounter{page}{-1}%
    \newpage}


\acsetup{first-style=short}

% Exemplos de acrônimos, se necessários...
% class `abbrev': abbreviations:
\DeclareAcronym{DFT}{
  short = DFT ,
  long  = \textit{Discrete Fourier Transform} ,
  class = abbrev
}
\DeclareAcronym{IDFT}{
  short = IDFT ,
  long  = \textit{Inverse Fourier Transform} ,
  class = abbrev
}
\DeclareAcronym{FFT}{
  short = FFT ,
  long  = \textit{Fast Fourier Transform} ,
  class = abbrev
}

\DeclareAcronym{IFFT}{
  short = IFFT ,
  long  = \textit{Inverse Fast Fourier Transform} ,
  class = abbrev
}
\DeclareAcronym{2D DFT}{
  short = 2D DFT ,
  long  = \textit{Two-Dimensional Discrete Fourier Transform} ,
  class = abbrev
}

\providecommand{\keywords}[1]{\textbf{\textit{Keywords: }} #1}
\providecommand{\palavrasChaves}[1]{\textbf{\textit{Palavras-chaves: }} #1}

\newdateformat{monthyeardate}{%
  \monthname[\THEMONTH], \THEYEAR}
%%%%%%%%%%%%%%%%%%%%%%%%%%%%%%%%%%%%%%%%%%%%%%%%%%%%%%%%%%%%%
\begin{document}

\selectlanguage{brazil}
\begin{titlepage}

	\centering
	{\normalsize \workauthor \par}
	%\includegraphics[width=0.15\textwidth]{example-image-1x1}\par\vspace{1cm}
	%{\scshape\LARGE Universidade Federal do Rio Grande do Norte \par}
	\vfill
	{\Large\bfseries \worktitle \par}
	\vfill

% Bottom of the page

	{\normalsize Brasil\par}
	{\normalsize \monthyeardate\today}
\end{titlepage}

\begin{titlepage}

	\centering
	{\normalsize \workauthor\par}
	%\includegraphics[width=0.15\textwidth]{example-image-1x1}\par\vspace{1cm}
	%{\scshape\LARGE Universidade Federal do Rio Grande do Norte \par}
	\vfill
	\centering
	{\Large\bfseries \par}
	\vfill

	\begin{flushright}	
	\begin{minipage}{15em}	
	\setstretch{1.0}
  	Trabalho de Conclusão de Curso Submetido à Coordenação do Curso de Engenharia de Computação e Automação do Centro de Tecnologia da Universidade Federal do Rio Grande do Norte, como parte dos requisitos necessários para a obtenção do grau de Engenheiro de Computação.
	\end{minipage}
	\end{flushright}	
	\vfill
	
	
	{\small Universidade Federal do Rio Grande do Norte - UFRN \par}
	{\small Coordenação do Curso de Engenharia de Computação e Automação - DCA \par}
	{\small Graduação em Engenharia de Computação \par}
	\vfill
	\normalsize
	\centering
	{\normalsize Orientador: Agostinho de Medeiros Brito Júnior \par}
	\vfill
% Bottom of the page
	{\normalsize Brasil\par}
	{\normalsize \monthyeardate\today}
\end{titlepage}

\pagenumbering{gobble}% Remove page numbers (and reset to 1)

%%%%%% AGRADECIMENTOS %%%%%%

\begin{center}
%{\bf \Large Agradecimentos}
\end{center}
%Gostaria de agradecer a...

\newpage

%%%%%% RESUMO %%%%%%

\begin{abstract}
  Apresenta o desenvolvimento de um {\it drum pad} usando visão
  artificial capaz de controlar sintetizadores musicais criando uma
  sequência de notas musicais em repetição. Instrumentos assim são
  muito usados por músicos amadores que precisam criar acompanhamentos
  de bateria ou baixo para suas composições e não dispõem de músicos
  auxiliares para fazê-lo. A ferramenta criada permite que usando
  apenas uma webcam, uma folha de papel e software livre um músico
  amador seja capaz de criar efeitos semelhantes aos de um drum pad
  físico desenhando ou sobrepondo pequenas fichas coloridas na folha
  de papel.
  \\
  \palavrasChaves{Drum pad, sequenciador, controlador midi, OpenCV,
    Visão artificial}
\end{abstract}

\newpage

%%%%%% RESUMO EM INGLÊS %%%%%%
\selectlanguage{english}

\begin{abstract}
  abstract in english
  \\
  \keywords{translate-as}

\end{abstract}

\newpage

\selectlanguage{brazil}

%%%%%% LISTA DE FIGURAS %%%%%%

\listoffigures

\newpage

%%%%%% LISTA DE ABREVIAÇÕES %%%%%%

%{\centering
%\printacronyms[include-classes=abbrev,name=Abreviações]
%}
\tableofcontents

%\makenomenclature
%\makeglossaries

%\newglossaryentry{DFT}{%
%name={DFT},%
%description={antigeen-presenterende cel}%
%}


\newpage

%%%%%% INÍCIO DO TEXTO %%%%%%

\pagenumbering{arabic}

\chapter{Introdução}
\label{cha:introducao}

Nesse capítulo, descreva o que é um drum pad, como funciona e porque
são tão importantes para músicos amadores. Fale sobre versatilidades e
limitações dos pads, custos de aquisição e como isso impacta nas
atividades de um músico amador.

Descreva como sua proposta pretende oferecer uma solução barata e de
fácil uso usando visão artificial, o que ela poderia oferecer quando
comparada à solução física e como softwares livres podem facilmente
interagir com a solução proposta.

No capítulo XX serão apresentados..., no capítulo XX será mostrado
... . O capítulo XX discorre sobre.... blá, blá, blá...

\chapter{Materiais e métodos}
\label{cha:fund-teor}

Descreva os principais algoritmos e tecnologias que utilizaste para
desenvolver a ferramenta, introduzindo a priori seu modelo de
PAD. Apresente a figura do pad, porque ela foi construída dessa forma
e descreva os algoritmos que escolheu para identificar as partes de
interesse no seu pad virtual. Nesse momento, ESQUEÇA que tudo foi
feito usando OpenCV, com destaque APENAS para os algoritmos.

Assuntos a serem abordados:

- O formato MIDI para comunicação entre Controlador e Sintetizador
(coloque uma seção só para isso. Quero aprender como funciona! :) )

- O processo de seleção de cores dos cantos da imagem. Salientar que
os marcadores existem para permitir a transformação de perspectiva.

- Descrever a MATEMÁTICA envolvida nessa transformação 

- Descrever COMO o usuário deve usar o quadriculado para selecionar as
notas a serem tocadas. A propósito, um pouquinho de teoria musical
explicando um bê-a-bá do uso do modelo é bem vinda. :)))

- Descrever o uso da classe RtMidi, porque foi escolhida e como se dá
sua utilização.

- Descrever o sintetizador usado nos experimentos (QSynth).

- Descrever como se dá, no Linux, a interligação entre seu software
Controlador e o Sintetizador. Um diagrama legal feito no inkscape
cairia bem nesse canto.

\chapter{Resultados}
\label{cha:resultados}

Apresentar exemplos de uso da ferramenta...

Tem um software bem legal de composição chamado rosegarden. Ele também
funciona como um ``sintetizador m MIDI'', pois aceita entradas do
controlador para permitir composições. Prepare um loop de exemplo e
conecte a saida do seu controlador na entrada do rosegarden. Observe a
sequencia gerada no software. Salve a sequencia e verifique se é
possível, usando o rosegarden, repetir a sequencia de loops conectando
agora rosegarden->qsynth.

Se funcionar legal (acredito que funcionará sem problemas), terás um
resultado muito bom, pois poderá mixar várias combinações possíveis
usando o rosegarden.

\chapter{Conclusões}
\label{cha:conclusoes}

Revise em linhas gerais o algoritmo desenvolvido e mostre os
progressos que obteve, comentando resultados e dificuldades
enfrentadas.

Proponha melhorias para a sua criação.

\bibliography{referencias}

\end{document}
